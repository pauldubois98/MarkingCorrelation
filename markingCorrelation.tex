\documentclass[a4paper]{article}
\usepackage[margin=2cm]{geometry}

%opening
\title{Marking Correlation}
\author{Romain Lhotte \& Paul Dubois\footnote{both authors contributed equally to this work}}

\begin{document}

\maketitle

\begin{abstract}
	The evaluation of students' performance in higher education plays a crucial role in assessing the effectiveness of teaching methods and improving learning outcomes. In this study, we investigate the correlation between two teachers' grading practices in a deep learning course at the master's level, offered at CentraleSupélec. The two teachers, who have distinct teaching styles and backgrounds, were responsible for marking different components of the course. We analyzed the grading distributions of the two teachers and assessed the correlation between their grades using Pearson's correlation coefficient. Our results indicate a significant positive correlation between the two teachers' grading practices, suggesting that their assessments of students' performance are consistent. Furthermore, we observed that the correlation was stronger for certain components of the course, such as the final project, compared to others. Our findings provide insight into the reliability and consistency of grading practices in higher education and can inform efforts to improve the quality of teaching and learning outcomes.
\end{abstract}

\section{}

\end{document}
