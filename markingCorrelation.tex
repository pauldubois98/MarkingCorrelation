\documentclass[a4paper]{article}
\usepackage[margin=2cm]{geometry}

%opening
\title{Marking Correlation}
\author{Romain Lhotte \& Paul Dubois\footnote{both authors contributed equally to this work}}

\begin{document}

\maketitle

\begin{abstract}
	The evaluation of students' performance in higher education plays a crucial role in assessing the effectiveness of teaching methods and improving learning outcomes.
	In this study, we investigate the correlation between two teachers' grading practices in a deep learning course at the master's level, offered at CentraleSupélec.
	The two teachers, who have distinct teaching styles and backgrounds, were responsible for marking the final project oral presentation.
	Our results indicate a significant positive correlation between the two teachers' grading practices, suggesting that their assessments of students' performance are consistent.
	Our findings provide insight into the reliability and consistency of grading practices in higher education and can inform efforts to improve the quality of teaching and learning outcomes.
\end{abstract}

\section{Introduction}
In recent years, there has been a surge of interest in the field of deep learning, with applications ranging from computer vision and natural language processing to bio-informatics and medical diagnosis.
As a result, there is an increasing demand for high-quality education and training programs in deep learning.
In response to this demand, many universities and engineering schools have started offering courses and programs in deep learning at various levels, including undergraduate and graduate levels.

This study focuses on the evaluation of a deep learning course offered to third-year engineering students at CentraleSupélec (Paris, France), who are pursuing the "Vivant, Santé, Environnement" (VSE) path.
The course was designed to provide students with a comprehensive understanding of the fundamental concepts, theories, and applications of deep learning.
The course was structured into 10 teaching sessions, each spanning three hours.
Each session comprised one hour of theoretical instruction followed by two hours of practical exercises.
In addition to the classroom instruction, five Kaggle challenges were assigned to the students, who were expected to work on them independently and outside of class time.
The final component of the course consisted of a group project, which required the 28 students to form 10 groups of 2-4 individuals.
The project served as a comprehensive assessment of the students' proficiency in deep learning and required the application of the concepts and techniques covered in the course.
Students were expected to present their project findings orally and respond to questions from the instructors.


\section{Methods}
\subsection{Participants}
The participants in this study were 28 students enrolled in a deep learning course at the master's level offered at CentraleSupélec during the academic year 2022/2023.
The class was taught by two instructors, both of whom had distinct teaching styles and backgrounds.

\subsection{Evaluation Framework}
The evaluation of students' performance in the course was based on six components, which accounted for the final grade.

The first five components were Kaggle challenges, each accounting for 6\% of the total mark, adding up to 30\% of the final mark.
For each challenge, a marking scheme was established, which was publicly shared with the students.
The marking scheme included specific scores to be reached on the leader-board, with 3/6, 4/6, 5/6, and 6/6 corresponding to different levels of performance on the challenges.
The top 10 students on the leader-board who had achieved at least 5/6 were awarded a score of 6/6 for the challenge.

The final component was a project, which accounted for the remaining 70\% of the final mark.
Students completed the project in groups of 2-4, and the project was assessed based on an oral presentation of approximately 10 minutes, followed by approximately 10 minutes of questions.
The final mark for the project was the average of the two independent marks given by the two instructors, who graded the projects separately.

\subsection{Ethical Considerations}
This study was conducted in accordance with the ethical guidelines of CentraleSupélec, and all participants were were anonymized for confidentiality purposes.
%The study was approved by the institutional review board of CentraleSupélec?

\section{Results}
\section{Discussion}
\section{Conclusion}

\end{document}
