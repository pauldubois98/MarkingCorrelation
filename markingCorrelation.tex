\documentclass[a4paper]{article}
\usepackage[margin=2cm]{geometry}

%opening
\title{Marking Correlation}
\author{Romain Lhotte \& Paul Dubois\footnote{both authors contributed equally to this work}}

\begin{document}

\maketitle

\begin{abstract}
	The evaluation of students' performance in higher education plays a crucial role in assessing the effectiveness of teaching methods and improving learning outcomes.
	In this study, we investigate the correlation between two teachers' grading practices in a deep learning course at the master's level, offered at CentraleSupélec.
	The two teachers, who have distinct teaching styles and backgrounds, were responsible for marking the final project oral presentation.
	Our results indicate a significant positive correlation between the two teachers' grading practices, suggesting that their assessments of students' performance are consistent.
	Our findings provide insight into the reliability and consistency of grading practices in higher education and can inform efforts to improve the quality of teaching and learning outcomes.
\end{abstract}

\section{Introduction}
\section{Methods}
\subsection{Participants}
The participants in this study were 28 students enrolled in a deep learning course at the master's level offered at CentraleSupélec during the academic year 2022/2023.
The class was taught by two instructors, both of whom had distinct teaching styles and backgrounds.
\subsection{Evaluation Framework}
\subsection{Ethical Considerations}
\section{Results}
\section{Discussion}
\section{Conclusion}

\end{document}
